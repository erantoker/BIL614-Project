
%% bare_conf.tex
%% V1.4
%% 2012/12/27
%% by Michael Shell
%% See:
%% http://www.michaelshell.org/
%% for current contact information.
%%
%% This is a skeleton file demonstrating the use of IEEEtran.cls
%% (requires IEEEtran.cls version 1.8 or later) with an IEEE conference paper.
%%
%% Support sites:
%% http://www.michaelshell.org/tex/ieeetran/
%% http://www.ctan.org/tex-archive/macros/latex/contrib/IEEEtran/
%% and
%% http://www.ieee.org/

%%*************************************************************************
%% Legal Notice:
%% This code is offered as-is without any warranty either expressed or
%% implied; without even the implied warranty of MERCHANTABILITY or
%% FITNESS FOR A PARTICULAR PURPOSE!
%% User assumes all risk.
%% In no event shall IEEE or any contributor to this code be liable for
%% any damages or losses, including, but not limited to, incidental,
%% consequential, or any other damages, resulting from the use or misuse
%% of any information contained here.
%%
%% All comments are the opinions of their respective authors and are not
%% necessarily endorsed by the IEEE.
%%
%% This work is distributed under the LaTeX Project Public License (LPPL)
%% ( http://www.latex-project.org/ ) version 1.3, and may be freely used,
%% distributed and modified. A copy of the LPPL, version 1.3, is included
%% in the base LaTeX documentation of all distributions of LaTeX released
%% 2003/12/01 or later.
%% Retain all contribution notices and credits.
%% ** Modified files should be clearly indicated as such, including  **
%% ** renaming them and changing author support contact information. **
%%
%% File list of work: IEEEtran.cls, IEEEtran_HOWTO.pdf, bare_adv.tex,
%%                    bare_conf.tex, bare_jrnl.tex, bare_jrnl_compsoc.tex,
%%                    bare_jrnl_transmag.tex
%%*************************************************************************


% Note that the a4paper option is mainly intended so that authors in
% countries using A4 can easily print to A4 and see how their papers will
% look in print - the typesetting of the document will not typically be
% affected with changes in paper size (but the bottom and side margins will).
% Use the testflow package mentioned above to verify correct handling of
% both paper sizes by the user's LaTeX system.
%
% Also note that the "draftcls" or "draftclsnofoot", not "draft", option
% should be used if it is desired that the figures are to be displayed in
% draft mode.
%
\documentclass[conference]{IEEEtran}
% Add the compsoc option for Computer Society conferences.
%
% If IEEEtran.cls has not been installed into the LaTeX system files,
% manually specify the path to it like:
% \documentclass[conference]{../sty/IEEEtran}





% Some very useful LaTeX packages include:
% (uncomment the ones you want to load)


% *** MISC UTILITY PACKAGES ***

% *** CITATION PACKAGES ***
%
\usepackage{cite}
% cite.sty was written by Donald Arseneau
% V1.6 and later of IEEEtran pre-defines the format of the cite.sty package
% \cite{} output to follow that of IEEE. Loading the cite package will
% result in citation numbers being automatically sorted and properly
% "compressed/ranged". e.g., [1], [9], [2], [7], [5], [6] without using
% cite.sty will become [1], [2], [5]--[7], [9] using cite.sty. cite.sty's
% \cite will automatically add leading space, if needed. Use cite.sty's
% noadjust option (cite.sty V3.8 and later) if you want to turn this off
% such as if a citation ever needs to be enclosed in parenthesis.
% cite.sty is already installed on most LaTeX systems. Be sure and use
% version 4.0 (2003-05-27) and later if using hyperref.sty. cite.sty does
% not currently provide for hyperlinked citations.
% The latest version can be obtained at:
% http://www.ctan.org/tex-archive/macros/latex/contrib/cite/
% The documentation is contained in the cite.sty file itself.






% *** GRAPHICS RELATED PACKAGES ***
%
\ifCLASSINFOpdf
  \usepackage[pdftex]{graphicx}
  % declare the path(s) where your graphic files are
  % \graphicspath{{../pdf/}{../jpeg/}}
  % and their extensions so you won't have to specify these with
  % every instance of \includegraphics
  % \DeclareGraphicsExtensions{.pdf,.jpeg,.png}
\else
  % or other class option (dvipsone, dvipdf, if not using dvips). graphicx
  % will default to the driver specified in the system graphics.cfg if no
  % driver is specified.
  \usepackage[dvips]{graphicx}
  % declare the path(s) where your graphic files are
  % \graphicspath{{../eps/}}
  % and their extensions so you won't have to specify these with
  % every instance of \includegraphics
  % \DeclareGraphicsExtensions{.eps}
\fi
% graphicx was written by David Carlisle and Sebastian Rahtz. It is
% required if you want graphics, photos, etc. graphicx.sty is already
% installed on most LaTeX systems. The latest version and documentation
% can be obtained at:
% http://www.ctan.org/tex-archive/macros/latex/required/graphics/
% Another good source of documentation is "Using Imported Graphics in
% LaTeX2e" by Keith Reckdahl which can be found at:
% http://www.ctan.org/tex-archive/info/epslatex/
%
% latex, and pdflatex in dvi mode, support graphics in encapsulated
% postscript (.eps) format. pdflatex in pdf mode supports graphics
% in .pdf, .jpeg, .png and .mps (metapost) formats. Users should ensure
% that all non-photo figures use a vector format (.eps, .pdf, .mps) and
% not a bitmapped formats (.jpeg, .png). IEEE frowns on bitmapped formats
% which can result in "jaggedy"/blurry rendering of lines and letters as
% well as large increases in file sizes.
%
% You can find documentation about the pdfTeX application at:
% http://www.tug.org/applications/pdftex



% correct bad hyphenation here
\hyphenation{op-tical net-works semi-conduc-tor}
\usepackage[utf8]{inputenc}
\usepackage[T1]{fontenc}


\begin{document}
%
% paper title
% can use linebreaks \\ within to get better formatting as desired
% Do not put math or special symbols in the title.
\title{Türk\c{c}e Haber Metinleri Üzerinden Popülerlik Tahmini}


% author names and affiliations
% use a multiple column layout for up to three different
% affiliations
\author{\IEEEauthorblockN{Eran Toker}
\IEEEauthorblockA{Hacettepe Üniversitesi\\
Ankara, Türkiye\\
Email: erantoker@gmail.com}
\and
\IEEEauthorblockN{Fatih Güler}
\IEEEauthorblockA{Hacettepe Üniversitesi\\
Ankara, Türkiye\\
Email: ffguler@gmail.com}
\and
\IEEEauthorblockN{Yi\u{g}it Sever}
\IEEEauthorblockA{Hacettepe Üniversitesi\\
Ankara, Türkiye\\
Email: yigitsever94@gmail.com}}


% make the title area
\maketitle

% As a general rule, do not put math, special symbols or citations
% in the abstract
\begin{abstract}
Haber kaynaklar{\i} her gün bir\c{c}ok haber makalesi yay{\i}nlamakta ancak yay{\i}nlanan bu metinlerin olduk\c{c}a kü\c{c}ük bir bölümü okuyucular{\i}n ilgisini \c{c}ekip popüler olabilmektedir. Popülerlik bir makalenin ald{\i}\u{g}{\i} t{\i}klanma say{\i}s{\i} arac{\i}l{\i}\u{g}{\i}yla öl\c{c}ülebilse de kullan{\i}c{\i}lar{\i}n ilgi duyduklar{\i} haberleri sosyal medya arac{\i}l{\i}\u{g}{\i}yla payla\c{s}{\i}p bir nevi bu metinlerin popülerli\u{g}i konusunda oy kullanabildikleri fikri göz önüne al{\i}nd{\i}\u{g}{\i}nda popülerli\u{g}i öl\c{c}mek i\c{c}in yenilik\c{c}i yöntemler kullan{\i}labilece\u{g}i anla\c{s}{\i}l{\i}r. Bu \c{c}al{\i}\c{s}mada, Milliyet.com.tr üzerinden payla\c{s}{\i}lan haberlerin, Türk\c{c}e sosyal medya ve tart{\i}\c{s}ma platformu Ek\c{s}iSözlük.com üzerinden payla\c{s}{\i}l{\i}p payla\c{s}{\i}lmayaca\u{g}{\i}n{\i} tahmin etmeye \c{c}al{\i}\c{s}{\i}yoruz.
\end{abstract}

\section{Giri\c{s}}

Sosyal medyan{\i}n, özellikle ki\c{s}isel bloglar ve canl{\i} yay{\i}n sayfalar{\i}n{\i}n istikrarl{\i} büyümesi, i\c{s}lenmeye haz{\i}r yeni veri türlerini de beraberinde getirdi. \c{c}evrimi\c{c}i tart{\i}\c{s}malar üzerinden kitap sat{\i}\c{s} rakamlar{\i}n{\i}n tahmin edilmesi \c{c}al{\i}\c{s}mas{\i} buna bir örnektir \cite{gruhl_predictive_2005}. Sosyal medya üzerinden payla\c{s}{\i}lan gönderilerin büyük \c{c}o\u{g}unlu\u{g}u kullan{\i}c{\i}lar{\i}n ya\c{s}ad{\i}klar{\i} olaylar üzerine gösterdikleri tepkileridir. Kullan{\i}c{\i}lar{\i}n olaylara eri\c{s}imi haber metinleri üzerinden oluyor ise bu konuyla ilgili payla\c{s}{\i}mlar{\i}nda ilgili haber metninin ba\u{g}lant{\i}s{\i}n{\i} da gönderilerine eklemeleri, ya da Ek\c{s}iSözlük.com örne\u{g}indeki gibi konu ile alakal{\i} yeni bir ba\c{s}l{\i}k alt{\i}nda konu\c{s}malar{\i} beklenebilir. Kullan{\i}c{\i}lar{\i}n bu davran{\i}\c{s}{\i}, sosyal medya uzmanlar{\i} i\c{c}in olaylar{\i}n kamuoyundaki yans{\i}malar{\i}n{\i} de\u{g}erlendirmek a\c{c}{\i}s{\i}ndan önemli bir kaynak ve yöntemdir.

Bizim bu \c{c}al{\i}\c{s}madaki amac{\i}m{\i}z bir haber makalesinin ne kadar popüler olaca\u{g}{\i}n{\i} haber yay{\i}nlanmadan önce tahmin etmeye \c{c}al{\i}\c{s}makt{\i}r. Makalenin popülerli\u{g}i tahmin etmek, o makalenin yay{\i}nlanmaya de\u{g}er olup olmad{\i}\u{g}{\i}n{\i}, ya da haberin alaca\u{g}{\i} reklamlar{\i}n de\u{g}erini öl\c{c}mede yard{\i}mc{\i} olabilir \cite{phukan_feasibility_2016}. E\u{g}er haber ajanslar{\i} hangi makalelerin popülerlik kazanaca\u{g}{\i}n{\i} önceden tahmin edebilirlerse kaynaklar{\i}n{\i} yüksek potansiyelli adaylar icin kullanabilirler. Bu noktada kullan{\i}c{\i}lara daha \c{c}ok ilgi duyabilecekleri makaleleri göstermenin kullan{\i}c{\i} deneyimi a\c{c}{\i}s{\i}ndan da önemi unutulmamal{\i}d{\i}r. E\u{g}er haber ajanslar{\i} kullan{\i}c{\i}lardan daha \c{c}ok t{\i}klama al{\i}rsa, daha \c{c}ok reklam geliri elde edecekleri a\c{s}ikard{\i}r. \cite{bandari_pulse_2012}

Bu \c{c}al{\i}\c{s}mada popülerlik tan{\i}m{\i}, belirli bir haber metninin Ek\c{s}iSözlük.com sitesi üzerinden tart{\i}\c{s}{\i}l{\i}p tart{\i}\c{s}{\i}lmayaca\u{g}{\i}yla belirlenir. Haberde konu edilen olay, Ek\c{s}iSözlük.com üzerinden yay{\i}nland{\i}\u{g}{\i} gün tart{\i}\c{s}{\i}lmaya ba\c{s}land{\i}ysa haber popüler olarak kabul edilmi\c{s}tir.

Bu \c{c}al{\i}\c{s}ma alanyaz{\i}na birka\c{c} katk{\i} yapmaktad{\i}r. Öncelikle, Milliyet.com.tr üzerindeki haber metinlerinin dinamiklerini incelemektedir. Haber metinlerinin popülerli\u{g}ini tahmin etme sorusuna ilk kez yan{\i}t aramaktad{\i}r. Haber yay{\i}nlanmadan önce halihaz{\i}rda var olan metinsel ve anlamsal özellikleri kullanarak bir haberin Ek\c{s}iSözlük.com sitesinde tart{\i}\c{s}{\i}l{\i}p tart{\i}\c{s}{\i}lmayaca\u{g}{\i}n{\i} sorgulamaktad{\i}r. Önerilen bu özelliklerin nas{\i}l de\u{g}erlendirilece\u{g}ini anlatmaktad{\i}r. Son olarak, hata \c{c}özümlemesi yaparak yöntemden do\u{g}an s{\i}n{\i}fland{\i}rma hatalar{\i}na sebepler aramaktad{\i}r.


\section{\.{I}lgili \c{c}al{\i}\c{s}malar}
\c{c}o\u{g}u popülerlik tahmini \c{c}al{\i}\c{s}mas{\i}, i\c{c}eri\u{g}in ilk ba\c{s}ta ald{\i}\u{g}{\i} yorumlar/payla\c{s}{\i}mlar üzerinde kuruludur \cite{szabo_predicting_2010}. Bu yöntemlerin \c{c}al{\i}\c{s}abilmesi i\c{c}in i\c{c}eri\u{g}in yay{\i}nlanmas{\i} gerekmektedir ve bu i\c{c}erik yay{\i}nlanmadan önce yap{\i}lan \c{c}özümlemenin getirdi\u{g}i art{\i} noktalar{\i} sa\u{g}layamaz.
2009 y{\i}l{\i}nda yap{\i}lan bir \c{c}al{\i}\c{s}man{\i}n \cite{tsagkias_predicting_2009} popülerlik tahmin etmek i\c{c}in 5 elemanl{\i} bir özellik seti vard{\i}r: yüzey (surface), birikimli (cumulative), metinsel (textual), anlamsal (semantic) ve ger\c{c}ek dünya (real world). Yüzey özelli\u{g}i i\c{c}erikte ge\c{c}en resimler ya da i\c{c}eri\u{g}e katk{\i}da bulunan yazarlar{\i}n say{\i}s{\i}d{\i}r. Birikimli özelli\u{g}i i\c{c}eri\u{g}in yay{\i}nlanma saatinde yay{\i}nlanan di\u{g}er i\c{c}eriklerin say{\i}s{\i}d{\i}r. Ger\c{c}ek dünya özelli\u{g}i i\c{c}eri\u{g}in yay{\i}nland{\i}\u{g}{\i} gün havan{\i}n güne\c{s}li mi yoksa bulutlu mu oldu\u{g}uyla ilgilenmektedir. Sonu\c{c} olarak, \c{c}al{\i}\c{s}mam{\i}zda metin madencili\u{g}i konu kapsam{\i}nda bulunan \emph{metinsel} ve \emph{anlamsal} özelliklerini kulland{\i}k.

\section{Veri ve Özellikler}
\subsection{Haberleri Tan{\i}mak}
\c{c}al{\i}\c{s}mam{\i}zda kulland{\i}\u{g}{\i}m{\i}z veri kümesi \c{s}ubat 2016 ile A\u{g}ustos 2016 aras{\i}nda Milliyet.com.tr sitesinde yay{\i}nlanan 449 haber metninden olu\c{s}maktad{\i}r. Bu 449 haber aras{\i}ndan 60 tanesi Ek\c{s}iSözlük.com üzerinden payla\c{s}{\i}lmad{\i}\u{g}{\i}/tart{\i}\c{s}{\i}lmad{\i}\u{g}{\i} i\c{c}in 'popüler de\u{g}il' olarak etiketlenmi\c{s}tir. Veri kümemizin say{\i}ca az olmas{\i}n{\i}n sebebi halihaz{\i}rda haber makalelerini Ek\c{s}iSözlük.com ba\c{s}l{\i}klar{\i}yla ba\u{g}da\c{s}t{\i}ran bir yöntem ya da veri kümesi olmamas{\i}d{\i}r. Bu sebeple veri kümemiz bizler taraf{\i}ndan el ile haz{\i}rlanm{\i}\c{s}t{\i}r. De\u{g}inmemiz gereken ba\c{s}ka bir nokta ise Ek\c{s}iSözlük.com'un di\u{g}er benzer sosyal medyalar (Twitter, Reddit) gibi payla\c{s}{\i}lan ba\u{g}lant{\i}lar alt{\i}nda yap{\i}lan yorumlar ile de\u{g}il kullan{\i}c{\i}lar taraf{\i}ndan se\c{c}ilen konu ba\c{s}l{\i}klar{\i} alt{\i}na girilen yorumlar ile \c{c}al{\i}\c{s}mas{\i}d{\i}r. Bu durum otomatik bir e\c{s}le\c{s}tirme ve veri seti olu\c{s}turma yöntemini olduk\c{c}a zor k{\i}lmaktad{\i}r. Se\c{c}ilen makaleler tarand{\i}ktan sonra i\c{c}erikleri HTML format{\i}ndandan \c{c}{\i}kart{\i}lm{\i}\c{s}t{\i}r. Haber metinleri bir ba\c{s}l{\i}k, alt ba\c{s}l{\i}k ve as{\i}l i\c{c}erikten olu\c{s}maktad{\i}r ve \c{c}al{\i}\c{s}mam{\i}zda bu k{\i}s{\i}mlar u\c{c} uca eklenerek kullan{\i}lm{\i}\c{s}t{\i}r.

Bu a\c{s}amada ilk ara\c{s}t{\i}rma sorumuzu; kullan{\i}c{\i}lar taraf{\i}ndan haber makaleleri i\c{c}in yaz{\i}lan yorumlar{\i}n dinami\u{g}i nedir olacakt{\i}r. Tsagkias ve ark. \cite{tsagkias_predicting_2009} göre 5 özellik bu konuda öne \c{c}{\i}kmaktad{\i}r. Biz projemizde bu 5 özellikten metin madencili\u{g}ini i\c{c}eren 2 tanesini kullan{\i}yoruz. Metinsel özellik külliyat{\i}m{\i}zdaki en yayg{\i}n 200 kelimeyi kullanmaktad{\i}r. Kelimeler, log-olas{\i}l{\i}k puanlar{\i}na göre, her haber i\c{c}in s{\i}ralanm{\i}\c{s}t{\i}r. Anlamsal özellik ise haber metninde ge\c{c}en ki\c{s}i, kurum/kurulu\c{s} ve yerlerin belirtilme say{\i}s{\i}d{\i}r.

\subsection{Özellikler ve Kullan{\i}mlar{\i}}
Metinsel özellik i\c{c}in öncelikle bütün haber metinlerinde en \c{c}ok yayg{\i}n olan 200 kelimeyi bulduk. Sonras{\i}nda Türk\c{c}e etkisiz kelimeleri \c{c}{\i}kartt{\i}k ve Resha Türk\c{c}e Stemmer \cite{resha-turkish-stemmer} kullanarak terimlerin köklerini \c{c}{\i}kartt{\i}k. Göz hesab{\i}yla tahmin ett\u{g}imiz üzere, "Ba\c{s}bakan", "Cumhurba\c{s}kan{\i}", "Futbol", "Polis", "Asker", "\c{s}ehit", "Patlama" ve "Terör" kelimeleri veri setini ald{\i}\u{g}{\i}m{\i}z zaman dilimine bak{\i}ld{\i}\u{g}{\i}nda en yayg{\i}n olan ve en \c{c}ok konu\c{s}ulan konular ile ili\c{s}kilidir.

Anlamsal özellik i\c{c}in DBPedia-Spotlight \cite{isem2013daiber} arac{\i}n{\i} kulland{\i}k. Bu ara\c{c} ile ki\c{s}i, kurum/kurulu\c{s} ve yer bilgilerini edindik. Sonras{\i}nda 'Random Forest' \cite{random-forest} algoritmas{\i}nda kullan{\i}lmak üzere 4 ayr{\i} veri sütunu olu\c{s}turduk; yayg{\i}n terimlerin s{\i}kl{\i}\u{g}{\i}, insan say{\i}s{\i}, kurum/kurulu\c{s} say{\i}s{\i} ve yer say{\i}s{\i}. Bu özelliklerle birlikte, haber metninin popüler olup olmad{\i}\u{g}{\i}n{\i} tahmin etmeye \c{c}al{\i}\c{s}t{\i}k. Bu ama\c{c} \c{c}er\c{c}evesinde G{\i}tHub kullan{\i}c{\i}s{\i} 'ironmanMa'n{\i}n algoritmas{\i}n{\i} kendi ihtiya\c{c}lar{\i}m{\i}za göre tekrar geli\c{s}tirdik.

\section{Deney Kurulumu}
Ek\c{s}iSözlük.com'un bir API sa\u{g}lamad{\i}\u{g}{\i}ndan daha önce de bahsetmi\c{s}tik. Bu sebeple üzerinde konu\c{s}ulan haberleri elle se\c{c}memiz gerekti. Sitede konu\c{s}ulan haberlerden olu\c{s}an kümemizi olu\c{s}turduktan sonra, haber metinlerini Milliyet.com'dan se\c{c}erek \c{c}{\i}kard{\i}k ve aylara göre düzenledik. Bu a\c{s}amadan sonra en yayg{\i}n 200 kelimeyi bulduk ve 'MostCommonTerms' dosyas{\i}na yazd{\i}k. Sonras{\i}nda olu\c{s}turdu\u{g}umuz 'TermFrequencies' dosyas{\i}na metinsel özelliklerden buldu\u{g}umuz sonu\c{c}lar{\i} yazd{\i}k. DBpedia Spotlight arac{\i}n{\i} kullanarak, 'SemanticVariables' dosyas{\i}n{\i} her ay i\c{c}in ayr{\i} ayr{\i} \c{c}ok izlekli bir yöntem izleyerek doldurduk. 3 \c{c}{\i}kt{\i} dosyas{\i} kullanarak 'TreeData' dosyalar{\i}n{\i} olu\c{s}turduk ve bu dosyalar{\i} al{\i}\c{s}t{\i}rma ve s{\i}na ad{\i}mlar{\i} i\c{c}in kullanarak 'TreeData' ve 'TestTreeData' dosyalar{\i}n{\i} olu\c{s}turduk. Son olarak bu dosyalar ile Random Forest algoritmas{\i}n{\i} \c{c}al{\i}\c{s}t{\i}rd{\i}k ve F1 skorumuzu ve do\u{g}ruluk hesab{\i}m{\i}z{\i} yapt{\i}k.

\section{Bulgular}

Deney sonras{\i}nda \c{c}{\i}kan sonu\c{c}lar Tablo \ref{prf1_lc}'de gösterildi\u{g}i gibi olup Bandari ve arkada\c{s}lar{\i}n{\i}n buldu\u{g}u 0.8'lik kesinlik sonucuna yakla\c{s}mas{\i} a\c{c}{\i}s{\i}nda umut vericidir \cite{bandari_pulse_2012}.


%%%%%%%%%%%%	LOWER CASE TEST RESULTS		%%%%%%%%%%%%
\begin{table}[!t]
	\caption{Kü\c{c}ük Harf Kullan{\i}lan Veri Kümesi Sonu\c{c}lar{\i}}
	\label{test_result_lc}
	\centering
	\begin{tabular}{|c|c|}
		\hline
		Ger\c{c}ek Pozitif & 33 \\
		\hline
		Yanl{\i}\c{s} Pozitif & 8 \\
		\hline
		Yanl{\i}\c{s} Negatif & 15 \\
		\hline
	\end{tabular}
\end{table}
%%%%%%%%%%%%	LOWER CASE TEST RESULTS		%%%%%%%%%%%%
%%%%%%%%%%%%	LOWER CASE P R F1		%%%%%%%%%%%%
\begin{table}[!t]
	\caption{Precision, Recall and F1 Scores for Lowercase dataset}
	\label{prf1_lc}
	\centering
	\begin{tabular}{|c|c|}
		\hline
		Kesinlik & 0.80 \\
		\hline
		Geri \c{c}a\u{g}{\i}rma & 0.69 \\
		\hline
		F1 Puan{\i} & 0.74 \\
		\hline
	\end{tabular}
\end{table}

%%%%%%%%%%%%	LOWER CASE P R F1		%%%%%%%%%%%%

Yukar{\i}da da belirtti\u{g}imiz üzere veri kümemizi deneyleri yapmadan önce kü\c{c}ük harfe \c{c}evirmi\c{s}tik.
Ancak DBSpotlight arac{\i}n{\i}n kü\c{c}ük harfe kar\c{s}{\i} duyarl{\i} oldu\u{g}unu ve kü\c{c}ük harfe \c{c}evrilen veri kümelerinde ki\c{s}i bilgisini ka\c{c}{\i}rmaya ba\c{s}lad{\i}\u{g}{\i}n{\i} fark ettik. Kü\c{c}ük harf kullan{\i}lmas{\i}n{\i}n yer ve kurum bilgisine bir etkisi olmad{\i}\u{g}{\i} da belirtilmelidir. Bu hatadan kurtulmak ad{\i}na veri kümemizi as{\i}l haline geri \c{c}evirerek deneyleri bir kez daha yapt{\i}k. Tablo \ref{test_result_sc} ve \ref{prf1_sc}'de gösterilen sonu\c{c}lar {\i}\c{s}{\i}\u{g}{\i}nda de\u{g}i\c{s}tirilmemi\c{s} harf senaryosunda \c{c}ok daha iyi sonu\c{c}lar al{\i}nd{\i}\u{g}{\i}n{\i} söyleyebiliriz.
%%%%%%%%%%%%	Sentence CASE TEST RESULTS		%%%%%%%%%%%%
\begin{table}[!t]
	\caption{De\u{g}i\c{s}tirilmemi\c{s} Harflerden Olu\c{s}turulan Veri K\"{u}mesi Deney Sonu\c{c}lar{\i}}
	\label{test_result_sc}
	\centering
	\begin{tabular}{|c|c|}
		\hline
		Gerçek Pozitif & 36 \\
		\hline
		Yanl{\i}\c{s} Pozitif & 7 \\
		\hline
		Yanl{\i}\c{s} Negatif & 12 \\
		\hline
	\end{tabular}
\end{table}
%%%%%%%%%%%%	Sentence CASE TEST RESULTS		%%%%%%%%%%%%
%%%%%%%%%%%%	Sentence CASE P R F1		%%%%%%%%%%%%
\begin{table}[!t]
	\caption{De\u{g}i\c{s}tirilmemi\c{s} Harflerden Olu\c{s}turulan Veri K\"{u}mesi Deneylerindeki Kesinlik, Geri \c{C}ekme ve F1 Puanlar{\i}}
	\label{prf1_sc}
	\centering
	\begin{tabular}{|c|c|}
		\hline
		Kesinlik & 0.83 \\
		\hline
		Geri \c{c}a\u{g}{\i}rma & 0.75 \\
		\hline
		F1 Puan{\i} & 0.79 \\
		\hline
	\end{tabular}
\end{table}

%%%%%%%%%%%%	Sentence CASE TEST RESULTS		%%%%%%%%%%%%

\section{Sonu\c{c}}
Bu \c{c}al{\i}\c{s}mada, haber makalelerinin olas{\i} popüleritelerini, henüz yay{\i}nlanmadan tahmin etmeyi ama\c{c}lad{\i}k. Bu ama\c{c}la verileri toplad{\i}k ve bu konuda daha önce yap{\i}lm{\i}\c{s} ara\c{s}t{\i}rmalara göre özellik setini \c{c}{\i}kard{\i}k. Algoritma ad{\i}mlar{\i}n{\i} i\c{s}lettikten sonra popüleritesi yanl{\i}\c{s} tahmin edilmi\c{s} makalelere göz att{\i}k. Bizim yöntemimize göre popüler olarak etiketlenmi\c{s} fakat ek\c{s}i sözlükte özel olarak ba\c{s}l{\i}\u{g}{\i} a\c{c}{\i}lmam{\i}\c{s} makalelerin, asl{\i}nda ger\c{c}ekten popüler oldu\u{g}u, fakat ba\c{s}l{\i}\u{g}{\i} a\c{c}{\i}lmadan makalede yer alan ki\c{s}i veya kurumlar{\i}n ba\c{s}l{\i}klar{\i}nda konunun tart{\i}\c{s}{\i}ld{\i}\u{g}{\i}n{\i} gözlemledik. Makalelerin yanl{\i}\c{s} pozitif olarak etiketlenmelerinin bir ba\c{s}ka sebebi ise, makalede bir \c{c}ok popüler terim/ki\c{s}i/kurum ge\c{c}mesine ra\u{g}men i\c{c}eriklerin bo\c{s} olmas{\i}yd{\i} (T{\i}k Tuza\u{g}{\i}).

% An example of a floating figure using the graphicx package.
% Note that \label must occur AFTER (or within) \caption.
% For figures, \caption should occur after the \includegraphics.
% Note that IEEEtran v1.7 and later has special internal code that
% is designed to preserve the operation of \label within \caption
% even when the captionsoff option is in effect. However, because
% of issues like this, it may be the safest practice to put all your
% \label just after \caption rather than within \caption{}.
%
% Reminder: the "draftcls" or "draftclsnofoot", not "draft", class
% option should be used if it is desired that the figures are to be
% displayed while in draft mode.
%


% Note that IEEE typically puts floats only at the top, even when this
% results in a large percentage of a column being occupied by floats.


% An example of a double column floating figure using two subfigures.
% (The subfig.sty package must be loaded for this to work.)
% The subfigure \label commands are set within each subfloat command,
% and the \label for the overall figure must come after \caption.
% \hfil is used as a separator to get equal spacing.
% Watch out that the combined width of all the subfigures on a
% line do not exceed the text width or a line break will occur.
%
%\begin{figure*}[!t]
%\centering
%\subfloat[Case I]{\includegraphics[width=2.5in]{box}%
%\label{fig_first_case}}
%\hfil
%\subfloat[Case II]{\includegraphics[width=2.5in]{box}%
%\label{fig_second_case}}
%\caption{Simulation results.}
%\label{fig_sim}
%\end{figure*}
%
% Note that often IEEE papers with subfigures do not employ subfigure
% captions (using the optional argument to \subfloat[]), but instead will
% reference/describe all of them (a), (b), etc., within the main caption.


% An example of a floating table. Note that, for IEEE style tables, the
% \caption command should come BEFORE the table. Table text will default to
% \footnotesize as IEEE normally uses this smaller font for tables.
% The \label must come after \caption as always.
%
%\begin{table}[!t]
%%% increase table row spacing, adjust to taste
%%\renewcommand{\arraystretch}{1.3}
%% if using array.sty, it might be a good idea to tweak the value of
%% \extrarowheight as needed to properly center the text within the cells
%\caption{An Example of a Table}
%\label{table_example}
%\centering
%%% Some packages, such as MDW tools, offer better commands for making tables
%%% than the plain LaTeX2e tabular which is used here.
%\begin{tabular}{|c||c|}
%\hline
%One & Two\\
%\hline
%Three & Four\\
%\hline
%\end{tabular}
%\end{table}

% Note that IEEE does not put floats in the very first column - or typically
% anywhere on the first page for that matter. Also, in-text middle ("here")
% positioning is not used. Most IEEE journals/conferences use top floats
% exclusively. Note that, LaTeX2e, unlike IEEE journals/conferences, places
% footnotes above bottom floats. This can be corrected via the \fnbelowfloat
% command of the stfloats package.


% references section

% can use a bibliography generated by BibTeX as a .bbl file
% BibTeX documentation can be easily obtained at:
% http://www.ctan.org/tex-archive/biblio/bibtex/contrib/doc/
% The IEEEtran BibTeX style support page is at:
% http://www.michaelshell.org/tex/ieeetran/bibtex/
\bibliographystyle{IEEEtran}
% argument is your BibTeX string definitions and bibliography database(s)
\bibliography{bibi}
%
% <OR> manually copy in the resultant .bbl file
% set second argument of \begin to the number of references
% (used to reserve space for the reference number labels box)
%\begin{thebibliography}{2}
%
%\bibitem{IEEEhowto:kopka}
%H.~Kopka and P.~W. Daly, \emph{A Guide to \LaTeX}, 3rd~ed.\hskip 1em plus
%  0.5em minus 0.4em\relax Harlow, England: Addison-Wesley, 1999.
%
%\end{thebibliography}




% that's all folks
\end{document}


