
%% bare_conf.tex
%% V1.4
%% 2012/12/27
%% by Michael Shell
%% See:
%% http://www.michaelshell.org/
%% for current contact information.
%%
%% This is a skeleton file demonstrating the use of IEEEtran.cls
%% (requires IEEEtran.cls version 1.8 or later) with an IEEE conference paper.
%%
%% Support sites:
%% http://www.michaelshell.org/tex/ieeetran/
%% http://www.ctan.org/tex-archive/macros/latex/contrib/IEEEtran/
%% and
%% http://www.ieee.org/

%%*************************************************************************
%% Legal Notice:
%% This code is offered as-is without any warranty either expressed or
%% implied; without even the implied warranty of MERCHANTABILITY or
%% FITNESS FOR A PARTICULAR PURPOSE!
%% User assumes all risk.
%% In no event shall IEEE or any contributor to this code be liable for
%% any damages or losses, including, but not limited to, incidental,
%% consequential, or any other damages, resulting from the use or misuse
%% of any information contained here.
%%
%% All comments are the opinions of their respective authors and are not
%% necessarily endorsed by the IEEE.
%%
%% This work is distributed under the LaTeX Project Public License (LPPL)
%% ( http://www.latex-project.org/ ) version 1.3, and may be freely used,
%% distributed and modified. A copy of the LPPL, version 1.3, is included
%% in the base LaTeX documentation of all distributions of LaTeX released
%% 2003/12/01 or later.
%% Retain all contribution notices and credits.
%% ** Modified files should be clearly indicated as such, including  **
%% ** renaming them and changing author support contact information. **
%%
%% File list of work: IEEEtran.cls, IEEEtran_HOWTO.pdf, bare_adv.tex,
%%                    bare_conf.tex, bare_jrnl.tex, bare_jrnl_compsoc.tex,
%%                    bare_jrnl_transmag.tex
%%*************************************************************************


% Note that the a4paper option is mainly intended so that authors in
% countries using A4 can easily print to A4 and see how their papers will
% look in print - the typesetting of the document will not typically be
% affected with changes in paper size (but the bottom and side margins will).
% Use the testflow package mentioned above to verify correct handling of
% both paper sizes by the user's LaTeX system.
%
% Also note that the "draftcls" or "draftclsnofoot", not "draft", option
% should be used if it is desired that the figures are to be displayed in
% draft mode.
%
\documentclass[conference]{IEEEtran}
% Add the compsoc option for Computer Society conferences.
%
% If IEEEtran.cls has not been installed into the LaTeX system files,
% manually specify the path to it like:
% \documentclass[conference]{../sty/IEEEtran}





% Some very useful LaTeX packages include:
% (uncomment the ones you want to load)


% *** MISC UTILITY PACKAGES ***

% *** CITATION PACKAGES ***
%
\usepackage{cite}
% cite.sty was written by Donald Arseneau
% V1.6 and later of IEEEtran pre-defines the format of the cite.sty package
% \cite{} output to follow that of IEEE. Loading the cite package will
% result in citation numbers being automatically sorted and properly
% "compressed/ranged". e.g., [1], [9], [2], [7], [5], [6] without using
% cite.sty will become [1], [2], [5]--[7], [9] using cite.sty. cite.sty's
% \cite will automatically add leading space, if needed. Use cite.sty's
% noadjust option (cite.sty V3.8 and later) if you want to turn this off
% such as if a citation ever needs to be enclosed in parenthesis.
% cite.sty is already installed on most LaTeX systems. Be sure and use
% version 4.0 (2003-05-27) and later if using hyperref.sty. cite.sty does
% not currently provide for hyperlinked citations.
% The latest version can be obtained at:
% http://www.ctan.org/tex-archive/macros/latex/contrib/cite/
% The documentation is contained in the cite.sty file itself.






% *** GRAPHICS RELATED PACKAGES ***
%
\ifCLASSINFOpdf
  \usepackage[pdftex]{graphicx}
  % declare the path(s) where your graphic files are
  % \graphicspath{{../pdf/}{../jpeg/}}
  % and their extensions so you won't have to specify these with
  % every instance of \includegraphics
  % \DeclareGraphicsExtensions{.pdf,.jpeg,.png}
\else
  % or other class option (dvipsone, dvipdf, if not using dvips). graphicx
  % will default to the driver specified in the system graphics.cfg if no
  % driver is specified.
  \usepackage[dvips]{graphicx}
  % declare the path(s) where your graphic files are
  % \graphicspath{{../eps/}}
  % and their extensions so you won't have to specify these with
  % every instance of \includegraphics
  % \DeclareGraphicsExtensions{.eps}
\fi
% graphicx was written by David Carlisle and Sebastian Rahtz. It is
% required if you want graphics, photos, etc. graphicx.sty is already
% installed on most LaTeX systems. The latest version and documentation
% can be obtained at:
% http://www.ctan.org/tex-archive/macros/latex/required/graphics/
% Another good source of documentation is "Using Imported Graphics in
% LaTeX2e" by Keith Reckdahl which can be found at:
% http://www.ctan.org/tex-archive/info/epslatex/
%
% latex, and pdflatex in dvi mode, support graphics in encapsulated
% postscript (.eps) format. pdflatex in pdf mode supports graphics
% in .pdf, .jpeg, .png and .mps (metapost) formats. Users should ensure
% that all non-photo figures use a vector format (.eps, .pdf, .mps) and
% not a bitmapped formats (.jpeg, .png). IEEE frowns on bitmapped formats
% which can result in "jaggedy"/blurry rendering of lines and letters as
% well as large increases in file sizes.
%
% You can find documentation about the pdfTeX application at:
% http://www.tug.org/applications/pdftex



% correct bad hyphenation here
\hyphenation{op-tical net-works semi-conduc-tor}
\usepackage[utf8]{inputenc}
\usepackage[T1]{fontenc}


\begin{document}
%
% paper title
% can use linebreaks \\ within to get better formatting as desired
% Do not put math or special symbols in the title.
\title{Türkçe Haber Metinleri Üzerinden Popülerlik Tahmini}


% author names and affiliations
% use a multiple column layout for up to three different
% affiliations
\author{\IEEEauthorblockN{Eran Toker}
\IEEEauthorblockA{Hacettepe Üniversitesi\\
Ankara, Türkiye\\
Email: erantoker@gmail.com}
\and
\IEEEauthorblockN{Fatih Güler}
\IEEEauthorblockA{Hacettepe Üniversitesi\\
Ankara, Türkiye\\
Email: ffguler@gmail.com}
\and
\IEEEauthorblockN{Yiğit Sever}
\IEEEauthorblockA{Hacettepe Üniversitesi\\
Ankara, Türkiye\\
Email: yigitsever94@gmail.com}}


% make the title area
\maketitle

% As a general rule, do not put math, special symbols or citations
% in the abstract
\begin{abstract}
Haber kaynakları her gün birçok haber makalesi yayınlamakta ancak yayınlanan bu metinlerin oldukça küçük bir bölümü okuyucuların ilgisini çekip popüler olabilmektedir. Popülerlik bir makalenin aldığı tıklanma sayısı aracılığıyla ölçülebilse de kullanıcıların ilgi duydukları haberleri sosyal medya aracılığıyla paylaşıp bir nevi bu metinlerin popülerliği konusunda oy kullanabildikleri fikri göz önüne alındığında popülerliği ölçmek için yenilikçi yöntemler kullanılabileceği anlaşılır. Bu çalışmada, Milliyet.com.tr üzerinden paylaşılan haberlerin, Türkçe sosyal medya ve tartışma platformu EkşiSözlük.com üzerinden paylaşılıp paylaşılmayacağını tahmin etmeye çalışıyoruz.
\end{abstract}

\section{Giriş}

Sosyal medyanın, özellikle kişisel bloglar ve canlı yayın sayfalarının istikrarlı büyümesi, işlenmeye hazır yeni veri türlerini de beraberinde getirdi. Çevrimiçi tartışmalar üzerinden kitap satış rakamlarının tahmin edilmesi çalışması buna bir örnektir \cite{gruhl_predictive_2005}. Sosyal medya üzerinden paylaşılan gönderilerin büyük çoğunluğu kullanıcıların yaşadıkları olaylar üzerine gösterdikleri tepkileridir. Kullanıcıların olaylara erişimi haber metinleri üzerinden oluyor ise bu konuyla ilgili paylaşımlarında ilgili haber metninin bağlantısını da gönderilerine eklemeleri, ya da EkşiSözlük.com örneğindeki gibi konu ile alakalı yeni bir başlık altında konuşmaları beklenebilir. Kullanıcıların bu davranışı, sosyal medya uzmanları için olayların kamuoyundaki yansımalarını değerlendirmek açısından önemli bir kaynak ve yöntemdir.

Bizim bu çalışmadaki amacımız bir haber makalesinin ne kadar popüler olacağını haber yayınlanmadan önce tahmin etmeye çalışmaktır. Makalenin popülerliği tahmin etmek, o makalenin yayınlanmaya değer olup olmadığını, ya da haberin alacağı reklamların değerini ölçmede yardımcı olabilir \cite{phukan_feasibility_2016}. Eğer haber ajansları hangi makalelerin popülerlik kazanacağını önceden tahmin edebilirlerse kaynaklarını yüksek potansiyelli adaylar icin kullanabilirler. Bu noktada kullanıcılara daha çok ilgi duyabilecekleri makaleleri göstermenin kullanıcı deneyimi açısından da önemi unutulmamalıdır. Eğer haber ajansları kullanıcılardan daha çok tıklama alırsa, daha çok reklam geliri elde edecekleri aşikardır. \cite{bandari_pulse_2012}

Bu çalışmada popülerlik tanımı, belirli bir haber metninin EkşiSözlük.com sitesi üzerinden tartışılıp tartışılmayacağıyla belirlenir. Haberde konu edilen olay, EkşiSözlük.com üzerinden yayınlandığı gün tartışılmaya başlandıysa haber popüler olarak kabul edilmiştir.

Bu çalışma alanyazına birkaç katkı yapmaktadır. Öncelikle, Milliyet.com.tr üzerindeki haber metinlerinin dinamiklerini incelemektedir. Haber metinlerinin popülerliğini tahmin etme sorusuna ilk kez yanıt aramaktadır. Haber yayınlanmadan önce halihazırda var olan metinsel ve anlamsal özellikleri kullanarak bir haberin EkşiSözlük.com sitesinde tartışılıp tartışılmayacağını sorgulamaktadır. Önerilen bu özelliklerin nasıl değerlendirileceğini anlatmaktadır. Son olarak, hata çözümlemesi yaparak yöntemden doğan sınıflandırma hatalarına sebepler aramaktadır.


\section{İlgili Çalışmalar}
Çoğu popülerlik tahmini çalışması, içeriğin ilk başta aldığı yorumlar/paylaşımlar üzerinde kuruludur \cite{szabo_predicting_2010}. Bu yöntemlerin çalışabilmesi için içeriğin yayınlanması gerekmektedir ve bu içerik yayınlanmadan önce yapılan çözümlemenin getirdiği artı noktaları sağlayamaz.
2009 yılında yapılan bir çalışmanın \cite{tsagkias_predicting_2009} popülerlik tahmin etmek için 5 elemanlı bir özellik seti vardır: yüzey (surface), birikimli (cumulative), metinsel (textual), anlamsal (semantic) ve gerçek dünya (real world). Yüzey özelliği içerikte geçen resimler ya da içeriğe katkıda bulunan yazarların sayısıdır. Birikimli özelliği içeriğin yayınlanma saatinde yayınlanan diğer içeriklerin sayısıdır. Gerçek dünya özelliği içeriğin yayınlandığı gün havanın güneşli mi yoksa bulutlu mu olduğuyla ilgilenmektedir. Sonuç olarak, çalışmamızda metin madenciliği konu kapsamında bulunan \emph{metinsel} ve \emph{anlamsal} özelliklerini kullandık.

\section{Veri ve Özellikler}
\subsection{Haberleri Tanımak}
Çalışmamızda kullandığımız veri kümesi Şubat 2016 ile Ağustos 2016 arasında Milliyet.com.tr sitesinde yayınlanan 449 haber metninden oluşmaktadır. Bu 449 haber arasından 60 tanesi EkşiSözlük.com üzerinden paylaşılmadığı/tartışılmadığı için 'popüler değil' olarak etiketlenmiştir. Veri kümemizin sayıca az olmasının sebebi halihazırda haber makalelerini EkşiSözlük.com başlıklarıyla bağdaştıran bir yöntem ya da veri kümesi olmamasıdır. Bu sebeple veri kümemiz bizler tarafından el ile hazırlanmıştır. Değinmemiz gereken başka bir nokta ise EkşiSözlük.com'un diğer benzer sosyal medyalar (Twitter, Reddit) gibi paylaşılan bağlantılar altında yapılan yorumlar ile değil kullanıcılar tarafından seçilen konu başlıkları altına girilen yorumlar ile çalışmasıdır. Bu durum otomatik bir eşleştirme ve veri seti oluşturma yöntemini oldukça zor kılmaktadır. Seçilen makaleler tarandıktan sonra içerikleri HTML formatındandan çıkartılmıştır. Haber metinleri bir başlık, alt başlık ve asıl içerikten oluşmaktadır ve çalışmamızda bu kısımlar uç uca eklenerek kullanılmıştır.

Bu aşamada ilk araştırma sorumuzu; kullanıcılar tarafından haber makaleleri için yazılan yorumların dinamiği nedir olacaktır. Tsagkias ve ark. \cite{tsagkias_predicting_2009} göre 5 özellik bu konuda öne çıkmaktadır. Biz projemizde bu 5 özellikten metin madenciliğini içeren 2 tanesini kullanıyoruz. Metinsel özellik külliyatımızdaki en yaygın 200 kelimeyi kullanmaktadır. Kelimeler, log-olasılık puanlarına göre, her haber için sıralanmıştır. Anlamsal özellik ise haber metninde geçen kişi, kurum/kuruluş ve yerlerin belirtilme sayısıdır.

\subsection{Özellikler ve Kullanımları}
Metinsel özellik için öncelikle bütün haber metinlerinde en çok yaygın olan 200 kelimeyi bulduk. Sonrasında Türkçe etkisiz kelimeleri çıkarttık ve Resha Türkçe Stemmer \cite{resha-turkish-stemmer} kullanarak terimlerin köklerini çıkarttık. Göz hesabıyla tahmin ettğimiz üzere, "Başbakan", "Cumhurbaşkanı", "Futbol", "Polis", "Asker", "Şehit", "Patlama" ve "Terör" kelimeleri veri setini aldığımız zaman dilimine bakıldığında en yaygın olan ve en çok konuşulan konular ile ilişkilidir.

Anlamsal özellik için DBPedia-Spotlight \cite{isem2013daiber} aracını kullandık. Bu araç ile kişi, kurum/kuruluş ve yer bilgilerini edindik. Sonrasında 'Random Forest' \cite{random-forest} algoritmasında kullanılmak üzere 4 ayrı veri sütunu oluşturduk; yaygın terimlerin sıklığı, insan sayısı, kurum/kuruluş sayısı ve yer sayısı. Bu özelliklerle birlikte, haber metninin popüler olup olmadığını tahmin etmeye çalıştık. Bu amaç çerçevesinde GıtHub kullanıcısı 'ironmanMa'nın algoritmasını kendi ihtiyaçlarımıza göre tekrar geliştirdik.

\section{Deney Kurulumu}
EkşiSözlük.com'un bir API sağlamadığından daha önce de bahsetmiştik. Bu sebeple üzerinde konuşulan haberleri elle seçmemiz gerekti. Sitede konuşulan haberlerden oluşan kümemizi oluşturduktan sonra, haber metinlerini Milliyet.com'dan seçerek çıkardık ve aylara göre düzenledik. Bu aşamadan sonra en yaygın 200 kelimeyi bulduk ve 'MostCommonTerms' dosyasına yazdık. Sonrasında oluşturduğumuz 'TermFrequencies' dosyasına metinsel özelliklerden bulduğumuz sonuçları yazdık. DBpedia Spotlight aracını kullanarak, 'SemanticVariables' dosyasını her ay için ayrı ayrı çok izlekli bir yöntem izleyerek doldurduk. 3 çıktı dosyası kullanarak 'TreeData' dosyalarını oluşturduk ve bu dosyaları alıştırma ve sına adımları için kullanarak 'TreeData' ve 'TestTreeData' dosyalarını oluşturduk. Son olarak bu dosyalar ile Random Forest algoritmasını çalıştırdık ve F1 skorumuzu ve doğruluk hesabımızı yaptık.

\section{Bulgular}

Deney sonrasında çıkan sonuçlar Tablo \ref{prf1_lc}'de gösterildiği gibi olup Bandari ve arkadaşlarının bulduğu 0.8'lik kesinlik sonucuna yaklaşması açısında umut vericidir \cite{bandari_pulse_2012}.


%%%%%%%%%%%%	LOWER CASE TEST RESULTS		%%%%%%%%%%%%
\begin{table}[!t]
	\caption{Küçük Harf Kullanılan Veri Kümesi Sonuçları}
	\label{test_result_lc}
	\centering
	\begin{tabular}{|c|c|}
		\hline
		Gerçek Pozitif & 33 \\
		\hline
		Yanlış Pozitif & 8 \\
		\hline
		Yanlıi Negatif & 15 \\
		\hline
	\end{tabular}
\end{table}
%%%%%%%%%%%%	LOWER CASE TEST RESULTS		%%%%%%%%%%%%
%%%%%%%%%%%%	LOWER CASE P R F1		%%%%%%%%%%%%
\begin{table}[!t]
	\caption{Precision, Recall and F1 Scores for Lowercase dataset}
	\label{prf1_lc}
	\centering
	\begin{tabular}{|c|c|}
		\hline
		Kesinlik & 0.80 \\
		\hline
		Geri Çağırma & 0.69 \\
		\hline
		F1 Puanı & 0.74 \\
		\hline
	\end{tabular}
\end{table}

%%%%%%%%%%%%	LOWER CASE P R F1		%%%%%%%%%%%%

Yukarıda da belirttiğimiz üzere veri kümemizi deneyleri yapmadan önce küçük harfe çevirmiştik.
Ancak DBSpotlight aracının küçük harfe karşı duyarlı olduğunu ve küçük harfe çevrilen veri kümelerinde kişi bilgisini kaçırmaya başladığını fark ettik. Küçük harf kullanılmasının yer ve kurum bilgisine bir etkisi olmadığı da belirtilmelidir. Bu hatadan kurtulmak adına veri kümemizi asıl haline geri çevirerek deneyleri bir kez daha yaptık. Tablo \ref{test_result_sc} ve \ref{prf1_sc}'de gösterilen sonuçlar ışığında değiştirilmemiş harf senaryosunda çok daha iyi sonuçlar alındığını söyleyebiliriz.
%%%%%%%%%%%%	Sentence CASE TEST RESULTS		%%%%%%%%%%%%
\begin{table}[!t]
	\caption{Test Results for Sentence Case dataset}
	\label{test_result_sc}
	\centering
	\begin{tabular}{|c|c|}
		\hline
		True Positive & 36 \\
		\hline
		False Positive & 7 \\
		\hline
		False Negative & 12 \\
		\hline
	\end{tabular}
\end{table}
%%%%%%%%%%%%	Sentence CASE TEST RESULTS		%%%%%%%%%%%%
%%%%%%%%%%%%	Sentence CASE P R F1		%%%%%%%%%%%%
\begin{table}[!t]
	\caption{Precision, Recall and F1 Scores for Sentence dataset}
	\label{prf1_sc}
	\centering
	\begin{tabular}{|c|c|}
		\hline
		Kesinlik & 0.83 \\
		\hline
		Geri Çağırma & 0.75 \\
		\hline
		F1 Puanı & 0.79 \\
		\hline
	\end{tabular}
\end{table}

%%%%%%%%%%%%	Sentence CASE TEST RESULTS		%%%%%%%%%%%%

\section{Sonuç}
Bu çalışmada, haber makalelerinin olası popüleritelerini, henüz yayınlanmadan tahmin etmeyi amaçladık. Bu amaçla verileri topladık ve bu konuda daha önce yapılmış araştırmalara göre özellik setini çıkardık. Algoritma adımlarını işlettikten sonra popüleritesi yanlış tahmin edilmiş makalelere göz attık. Bizim yöntemimize göre popüler olarak etiketlenmiş fakat ekşi sözlükte özel olarak başlığı açılmamış makalelerin, aslında gerçekten popüler olduğu, fakat başlığı açılmadan makalede yer alan kişi veya kurumların başlıklarında konunun tartışıldığını gözlemledik. Makalelerin yanlış pozitif olarak etiketlenmelerinin bir başka sebebi ise, makalede bir çok popüler terim/kişi/kurum geçmesine rağmen içeriklerin boş olmasıydı (Tık Tuzağı).

% An example of a floating figure using the graphicx package.
% Note that \label must occur AFTER (or within) \caption.
% For figures, \caption should occur after the \includegraphics.
% Note that IEEEtran v1.7 and later has special internal code that
% is designed to preserve the operation of \label within \caption
% even when the captionsoff option is in effect. However, because
% of issues like this, it may be the safest practice to put all your
% \label just after \caption rather than within \caption{}.
%
% Reminder: the "draftcls" or "draftclsnofoot", not "draft", class
% option should be used if it is desired that the figures are to be
% displayed while in draft mode.
%


% Note that IEEE typically puts floats only at the top, even when this
% results in a large percentage of a column being occupied by floats.


% An example of a double column floating figure using two subfigures.
% (The subfig.sty package must be loaded for this to work.)
% The subfigure \label commands are set within each subfloat command,
% and the \label for the overall figure must come after \caption.
% \hfil is used as a separator to get equal spacing.
% Watch out that the combined width of all the subfigures on a
% line do not exceed the text width or a line break will occur.
%
%\begin{figure*}[!t]
%\centering
%\subfloat[Case I]{\includegraphics[width=2.5in]{box}%
%\label{fig_first_case}}
%\hfil
%\subfloat[Case II]{\includegraphics[width=2.5in]{box}%
%\label{fig_second_case}}
%\caption{Simulation results.}
%\label{fig_sim}
%\end{figure*}
%
% Note that often IEEE papers with subfigures do not employ subfigure
% captions (using the optional argument to \subfloat[]), but instead will
% reference/describe all of them (a), (b), etc., within the main caption.


% An example of a floating table. Note that, for IEEE style tables, the
% \caption command should come BEFORE the table. Table text will default to
% \footnotesize as IEEE normally uses this smaller font for tables.
% The \label must come after \caption as always.
%
%\begin{table}[!t]
%%% increase table row spacing, adjust to taste
%%\renewcommand{\arraystretch}{1.3}
%% if using array.sty, it might be a good idea to tweak the value of
%% \extrarowheight as needed to properly center the text within the cells
%\caption{An Example of a Table}
%\label{table_example}
%\centering
%%% Some packages, such as MDW tools, offer better commands for making tables
%%% than the plain LaTeX2e tabular which is used here.
%\begin{tabular}{|c||c|}
%\hline
%One & Two\\
%\hline
%Three & Four\\
%\hline
%\end{tabular}
%\end{table}

% Note that IEEE does not put floats in the very first column - or typically
% anywhere on the first page for that matter. Also, in-text middle ("here")
% positioning is not used. Most IEEE journals/conferences use top floats
% exclusively. Note that, LaTeX2e, unlike IEEE journals/conferences, places
% footnotes above bottom floats. This can be corrected via the \fnbelowfloat
% command of the stfloats package.


% references section

% can use a bibliography generated by BibTeX as a .bbl file
% BibTeX documentation can be easily obtained at:
% http://www.ctan.org/tex-archive/biblio/bibtex/contrib/doc/
% The IEEEtran BibTeX style support page is at:
% http://www.michaelshell.org/tex/ieeetran/bibtex/
\bibliographystyle{IEEEtran}
% argument is your BibTeX string definitions and bibliography database(s)
\bibliography{bibi}
%
% <OR> manually copy in the resultant .bbl file
% set second argument of \begin to the number of references
% (used to reserve space for the reference number labels box)
%\begin{thebibliography}{2}
%
%\bibitem{IEEEhowto:kopka}
%H.~Kopka and P.~W. Daly, \emph{A Guide to \LaTeX}, 3rd~ed.\hskip 1em plus
%  0.5em minus 0.4em\relax Harlow, England: Addison-Wesley, 1999.
%
%\end{thebibliography}




% that's all folks
\end{document}


